\documentclass[12pt]{article}
%
\usepackage[utf8]{vietnam}
\usepackage{amscd,amsmath,amstext,amsfonts,amsbsy,amssymb,amsthm}
\usepackage{multicol,graphicx,enumerate}
\usepackage{array,multirow,longtable,slashbox}
\usepackage{fancyhdr,booktabs}
\usepackage[authoryear,round,longnamesfirst]{natbib}
%
\usepackage[unicode,colorlinks,linkcolor = blue,citecolor = blue]{hyperref}
\usepackage{color,alltt}
\usepackage{authblk}
\usepackage[ddmmyyyy,hhmmss]{datetime}
\usepackage{rotating}
%\usepackage{enumitem}
%
\usepackage[margin = 2.5cm]{geometry}
% setting up for caption of tables and figures
\usepackage[width=.85\textwidth, font = small, labelfont = bf, labelsep = endash, textfont = it]{caption}

\newtheorem{theorem}{{\bf Theorem}}%[section]
\theoremstyle{definition} \newtheorem{exercise}[theorem]{\bf Bài tập}

\setlength{\parskip}{6pt}
\baselineskip = 15pt
\setlength{\headheight}{15pt}

\newcommand{\x}{\mathnormal{x}}
\newcommand{\y}{\mathnormal{y}}
\newcommand{\f}{\mathnormal{f}}
\DeclareMathOperator*{\argmax}{arg\,max}
\DeclareMathOperator*{\argmin}{arg\,min}

\begin{document}
	
\begin{center}
	\textbf{\Large Trực quan hóa dữ liệu} \\
	\vspace{0.5cm}
	\textbf{-- Bài kiểm tra Power BI 2 --}
\end{center}

\vspace{0.5cm}

\noindent
\textbf{Yêu cầu:} hãy sử dụng các biểu đồ trực quan dữ liệu được học:
\begin{itemize}
\item biểu đồ phân tán;
\item biểu đồ tương quan;
\item biểu đồ chuỗi thời gian;
\item biểu đồ xu hướng (làm trơn, dạng hàm xác định);
\item biểu đồ với khoảng tin cậy.
\end{itemize}
để tạo thành 1 dashboard chứa các biểu đồ trực quan tối đa các tính chất của \textbf{một} trong các bộ dữ liệu được miêu tả dưới đây. Đồng thời, dashboard cũng cần thể hiện được tính tương tác với người dùng thông qua các bảng lọc, tách dữ liệu.

\noindent
\textbf{Chú ý:} chọn 1 trong số các dữ liệu dưới đây.

\section*{Data 1 - Video Game Sales}
Dữ liệu \texttt{vgsales.csv} cung cấp thông tin doanh thu của các trò chơi điện tử có doanh số bán ra lớn hơn 100.000 bản, từ 1980 tới 2020. Bao gồm các biến được quan sát như sau:
\begin{itemize}
\item \texttt{Rank}: thứ hạng theo doanh số tổng quát
\item \texttt{Name}: tên của trò chơi
\item \texttt{Platform}: nền tảng phát hành trò chơi (ví dụ: PC, PS4, ...)
\item \texttt{Year}: năm phát hành
\item \texttt{Genre}: Thể loại trò chơi
\item \texttt{Publisher}: nhà phát hành trò chơi
\item \texttt{NA\_Sales}: doanh số bán hàng tại Bắc Mỹ (đơn vị triệu \$)
\item \texttt{EU\_Sales}: doanh số bán hàng tại Châu Âu (đơn vị triệu \$)
\item \texttt{JP\_Sales}: doanh số bán hàng tại Nhật bản (đơn vị triệu \$)
\item \texttt{Other\_Sales}: doanh số bán hàng tại các quốc gia khác (đơn vị triệu \$)
\item \texttt{Global\_Sales}: tổng doanh số bán hàng trên toàn cầu (đơn vị triệu \$)
\end{itemize}
Có tổng cộng 16,598 trò chơi điện tử khác nhau được ghi chép lại.

\section*{Data 2 - E-Commerce Business Performance}
Kinh doanh thương mại điện tử (E-Commerce Business) bắt đầu hình từ những năm 1970 tại Mỹ, và liên tục cập nhật, phát triển cho tới ngày nay, trong đó, giai đoạn từ 2011, đánh dấu sự bùng nổ mạnh mẽ của kinh doanh thương mại điện tử kết hợp với các công nghệ mới. Dữ liệu \texttt{2\_Ecommerce Sales Data Analysis Excel.xlsx} chứa các thông tin về danh sách tất cả các giao dịch mua bàn hàng hóa trên các trang kinh doanh thương mại điện tử tại Mỹ, từ năm 2011 tới 2014, cùng với các chi tiết như - địa chỉ của khách hàng, loại dịch vụ vận chuyển, danh mục hàng hóa, tên mặt hàng, số lượng, doanh thu, lợi nhuận, v. v. cụ thể
\begin{itemize}
\item \texttt{Row ID}: mã số xác định duy nhất cho 1 ghi chép;
\item \texttt{Order ID}: mã số xác định đơn hàng;
\item \texttt{Year}: năm;
\item \texttt{Order Date}: ngày đặt hàng;
\item \texttt{Ship Date}: ngày chuyển hàng;
\item \texttt{shipment days}: số ngày chuyển hàng;
\item \texttt{Ship Mode}: loại dịch vụ chuyển hàng;
\item \texttt{Customer ID}: mã số xác định của khách hàng;
\item \texttt{Customer Name}: tên của khách hàng; 
\item \texttt{Segment}: Phân mục khách hàng;
\item \texttt{Country}: Quốc gia;
\item \texttt{City}: Thành phố;
\item \texttt{State}: Bang;
\item \texttt{Postal Code}: mã bưu cục;
\item \texttt{Region}: khu vực địa lý;
\item \texttt{Product ID}: mã sãn phẩm;
\item \texttt{Category}: danh mục hàng hóa;
\item \texttt{Sub-Category}: danh mục con của hàng hóa;
\item \texttt{Product Name}: tên sản phẩm;
\item \texttt{Sales}: doanh thu;
\item \texttt{Quantity}: số lượng mua;
\item \texttt{Discount}: phần trăm giảm giá;
\item \texttt{Profit}: lợi nhuận.
\end{itemize}

\section*{Data 3 - Fashion E-commerce sales}
Bộ dữ liệu \texttt{Fashion\_Sales\_Analysis.xlsx} bao gồm các đơn hàng do khách hàng đặt trên nhiều nền tảng thương mại điện tử khác nhau, bao gồm Myntra, Ajio, Amazon, Flipkart, Nykaa, Meesho và các nền tảng khác. Bộ dữ liệu bao gồm nhiều mặt hàng thời trang, bao gồm các danh mục như quần jean, giày dép, váy dân tộc, kurta, sari, bộ, áo và váy phương Tây, phục vụ cho cả nam và nữ, cụ thể:
\begin{itemize}
\item \texttt{Index}: mã định danh duy nhất cho mỗi hàng trong tập dữ liệu;
\item \texttt{Order ID}: mã định danh duy nhất cho mỗi đơn hàng đã đặt;
\item \texttt{Cust ID}: mã định danh duy nhất cho mỗi khách hàng;
\item \texttt{Gender}: giới tính của khách hàng;
\item \texttt{Age}: tuổi của khách hàng;
\item \texttt{Date}: ngày đặt hàng;
\item \texttt{Status}: trạng thái đơn hàng (ví dụ: Cancelled, Delivered, Returned, Refunded);
\item \texttt{Channel}: nền tảng thương mại điện tử được sử dụng để mua hàng (ví dụ: Amazon, Flipkart);
\item \texttt{SKU}: đơn vị lưu kho (mã định danh duy nhất cho mỗi sản phẩm)
\item \texttt{Category}: danh mục sản phẩm (ví dụ: quần jean, giày dép, sari);
\item \texttt{Size}: kích thước của sản phẩm;
\item \texttt{Qty}: số lượng sản phẩm đã đặt hàng;
\item \texttt{Amount}: tổng số tiền của đơn hàng;
\item \texttt{Ship-city}: thành phố mà đơn hàng được chuyển đến;
\item \texttt{Ship-state}: tiểu bang mà đơn hàng được chuyển đến;
\item \texttt{Ship-postal-code}: mã bưu chính của địa chỉ giao hàng;
\item \texttt{B2B}: chỉ ra liệu đơn hàng có phải là đơn hàng doanh nghiệp với doanh nghiệp (B2B) hay không.
\end{itemize}

\end{document}